\documentclass{leadsheet}
\usepackage[margin=1.5cm]{geometry}
%\usepackage{leadsheets}[bar-shortcuts=true]
\usepackage{multicol}
\usepackage{array}
\usepackage[german]{babel}
% use Helvetica
\usepackage{helvet}
\renewcommand{\familydefault}{\sfdefault}
\fontfamily{phv}\selectfont
\renewcommand{\normalsize}{\fontsize{15pt}{14pt}\selectfont}

\useleadsheetslibrary{musejazz}
\selectlanguage{german}
\DeclareTranslation{German}{leadsheets/interlude}{Zwischenspiel}


\usepackage{enumitem}
\SetLabelAlign{labelontop}%
{\strut\smash{\parbox[t]{\textwidth}{\quad#1}}}%

\makeatletter
\defineversetypetemplate {toplabel}
{%
  \ifversenamed{%
    \begin{enumerate}[align=labelontop,%
      leftmargin=1ex]%\enumerate
    }{%
    \begin{enumerate}[leftmargin=1ex]%  
    }%
  \ifobeylines%
  {%
    \setlength{\parskip}{0pt}%
    \setleadsheets{ obey-lines-parskip = \parsep }%
  }
  {}%
  \item[{\verselabel}]%
}
{\end{enumerate}}
\makeatother

\definesongtitletemplate{ourtitle}{
  %\ifsongmeasuring
      %{\section*}
      %{\section}%
      %{\songproperty{title}}
    \section*{\songproperty{title}}  
    \begingroup\footnotesize
    \begin{tabular}{
        @{}
        >{\raggedright\arraybackslash}p{.5\linewidth}
        @{}
        >{\raggedleft\arraybackslash}p{.5\linewidth}
        @{}
      }
      \ifsongproperty{interpret}
        {\GetTranslation{leadsheets/interpret}}
        {}%
      \ifsongproperty{composer}
        {%
          &
          \GetTranslation{leadsheets/composer}: %
          \printsongpropertylist{composer}{ \& }{, }{ \& }
          \ifsongproperty{lyrics}
            {
              \\ &
              \GetTranslation{leadsheets/lyrics}: %
              \printsongpropertylist{lyrics}{ \& }{, }{ \& }
            }
            {}%
        }
        {}%
      \ifsongproperty{interpret}{\\}{\ifsongproperty{composer}{\\}{}}%
      \ifsongproperty{genre}
        {& Genre: \songproperty{genre} \\}
        {}%
      \ifsongproperty{tempo}
        {& Tempo: \songproperty{tempo} \\}
        {}%
      \ifsongproperty{key}
        {%
          & \setchords{
              major = -\GetTranslation{leadsheets/major} ,
              minor = -\GetTranslation{leadsheets/minor}
            }%
          \GetTranslation{leadsheets/key}: %
          \expandcode{\writechord{\songproperty{key}}} \\%
        }
        {}%
    \end{tabular}
    \par\endgroup
  }


\begin{document}
\setleadsheets{
  align-chords={l} 
  , verse/template=toplabel
  , chorus/template=toplabel
  , chorus/label-format={\textbf}
  , interlude/template=toplabel
  , interlude/label-format={\textbf}
  % ,print-chords=false
}
\setsbfontsize{14pt}

\begin{song}[verse/numbered,remember-chords]
  { , title={Bunt sind schon die Wälder}
    , key=G
    , composer={Johann Friedrich Reichart 1799}
 %   , poet={Johann Gaudenz von Salis-Seewis 1793}
  }
  \begin{multicols}{2}

  \begin{verse}
  ^{G}Bunt sind schon die ^*{D}Wäl ^{G}der, \\
  gelb die Stoppel^*{D7}fel ^{G}der, \\
  ^{D}und der ^{A7}Herbst be^{D}ginnt. \\
  ^{G}Rote Blätter ^{C}fallen, \\
  ^{Am}graue Nebel ^{D7}wallen, \\
  ^{G}kühler ^{D7}weht der ^{G}Wind. \\
  \end{verse}
  \begin{verse}
  ^Wie die volle ^*Trau ^be  \\
  aus dem Reben^*lau ^be \\
  ^*purpur ^farbig ^strahlt!  \\
  ^Am Geländer ^reifen \\
  ^Pfirsiche, mit ^Streifen \\
  ^rot und ^weiß be^malt. \\
  \end{verse}
  \begin{verse}
  ^Sieh, wie hier die ^*Dir ^ne \\
  emsig Pflaum und ^*Bir ^ne \\
  ^in ihr ^Körbchen ^legt, \\
  ^dort mit leichten ^Schritten \\
  ^jene goldnen ^Quitten \\
  ^in den ^Landhof ^trägt. \\
  \end{verse}
\columnbreak
  \begin{verse}
   ^Flinke Träger ^*sprin ^gen, \\
  und die Mädchen ^*sin ^gen, \\
  ^alles ^jubelt ^froh! \\
  ^Bunte Bänder ^schweben \\
  ^zwischen hohen ^Reben \\
  ^auf dem ^Hut von ^Stroh. \\
  \end{verse}
  \begin{verse}
  ^Geige tönt und ^*Flö ^te \\
  bei der Abend^*rö ^te \\
  ^und im ^*Mondes ^glanz; \\
  ^junge Winze^rinnen \\
  ^winken und be^ginnen \\
  ^frohen ^*Ernte ^tanz. \\
  \end{verse}
    % \ifsbprintchords{\columnbreak}{}

  \end{multicols}
\end{song}
\end{document}




  


 
