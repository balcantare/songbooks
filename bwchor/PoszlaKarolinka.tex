\documentclass{leadsheet}
\usepackage[margin=1.5cm]{geometry}
%\usepackage{leadsheets}[bar-shortcuts=true]
\usepackage{multicol}
\usepackage{array}
\usepackage[german]{babel}
% use Helvetica
\usepackage{helvet}
\renewcommand{\familydefault}{\sfdefault}
\fontfamily{phv}\selectfont
\renewcommand{\normalsize}{\fontsize{15pt}{14pt}\selectfont}

\useleadsheetslibrary{musejazz}
\selectlanguage{german}
\DeclareTranslation{German}{leadsheets/interlude}{Zwischenspiel}


\usepackage{enumitem}
\SetLabelAlign{labelontop}%
{\strut\smash{\parbox[t]{\textwidth}{\quad#1}}}%

\makeatletter
\defineversetypetemplate {toplabel}
{%
  \ifversenamed{%
    \begin{enumerate}[align=labelontop,%
      leftmargin=1ex]%\enumerate
    }{%
    \begin{enumerate}[leftmargin=1ex]%  
    }%
  \ifobeylines%
  {%
    \setlength{\parskip}{0pt}%
    \setleadsheets{ obey-lines-parskip = \parsep }%
  }
  {}%
  \item[{\verselabel}]%
}
{\end{enumerate}}
\makeatother

\definesongtitletemplate{ourtitle}{
  %\ifsongmeasuring
      %{\section*}
      %{\section}%
      %{\songproperty{title}}
    \section*{\songproperty{title}}  
    \begingroup\footnotesize
    \begin{tabular}{
        @{}
        >{\raggedright\arraybackslash}p{.5\linewidth}
        @{}
        >{\raggedleft\arraybackslash}p{.5\linewidth}
        @{}
      }
      \ifsongproperty{interpret}
        {\GetTranslation{leadsheets/interpret}}
        {}%
      \ifsongproperty{composer}
        {%
          &
          \GetTranslation{leadsheets/composer}: %
          \printsongpropertylist{composer}{ \& }{, }{ \& }
          \ifsongproperty{lyrics}
            {
              \\ &
              \GetTranslation{leadsheets/lyrics}: %
              \printsongpropertylist{lyrics}{ \& }{, }{ \& }
            }
            {}%
        }
        {}%
      \ifsongproperty{interpret}{\\}{\ifsongproperty{composer}{\\}{}}%
      \ifsongproperty{genre}
        {& Genre: \songproperty{genre} \\}
        {}%
      \ifsongproperty{tempo}
        {& Tempo: \songproperty{tempo} \\}
        {}%
      \ifsongproperty{key}
        {%
          & \setchords{
              major = -\GetTranslation{leadsheets/major} ,
              minor = -\GetTranslation{leadsheets/minor}
            }%
          \GetTranslation{leadsheets/key}: %
          \expandcode{\writechord{\songproperty{key}}} \\%
        }
        {}%
    \end{tabular}
    \par\endgroup
  }


\begin{document}
\setleadsheets{
  align-chords={l} 
  , verse/template=toplabel
  , interlude/template=toplabel
  , interlude/named=false
  , interlude/after-label=
  % ,print-chords=false
}
\setsbfontsize{13pt}

\begin{song}[verse/numbered,remember-chords]
  { , title={Poszła Karolinka}
    , key=C
  }
  \begin{multicols}{2}

  \begin{verse}
  |: ^{G}Poszła Karolinka ^{C}do Gogoli^{G}na, :| \\
  |: ^{G}a~Karliczek za ^{C}nia, 
     ^{D}a~Karliczek za ^{G}nia, \\
     z~flaszecz^{D7}ka wi^{G}na. :| {\itshape *Ak} \\
  \end{verse}

  \begin{verse} 
  |: ^Karolinka geht nach ^Gogolin al^lein, :| \\
  ^Karlitschek folgt ei^lig, \\
  ^Karlitschek folgt ei^lig, \\
  mit 'nem ^Fläschchen ^Wein. {\itshape *Md} \\
  \end{verse}

  \begin{verse}
  |: ^Gogolin entgegen \\ ^wendet sie den ^Blick. :| \\
  ^Nach dem hübschen Bur^schen, \\
  ^nach dem hübschen Bur^schen, \\
  schaut sie ^nicht zu^rück. \\
  \end{verse}

  \begin{interlude}%[format={\itshape}]
    |: ^{G}Karolinka, Karo^{Am}linka, \\
    ^{D}warum gehst do so weit ^{G}fort? :| {\itshape *Ak} \\
  \end{interlude}
  
  \begin{verse}
    |: ^Liebes Weglein, führ mich \\ 
    ^in die weite ^Welt, :| \\
    ^führ mich zu dem Bur^schen, \\
    ^führ mich zu dem Bur^schen, \\
    der mir ^mehr ge^fällt. {\itshape *Md} \\
  \end{verse}
  \columnbreak
  \begin{verse}
    |: ^Laufe mir nicht nach, du, \\
    ^sag's dir jeden ^Tag, :| \\
    ^sag's dir immer wie^der, \\
    ^sag's dir immer wie^der, \\
    daß ich ^dich nicht ^mag. \\
  \end{verse}

  \begin{interlude}
   |: ^Karolinka, Karo^linka, \\
   ^was hab ich dir nur ge^tan? :| {\itshape *Ak} \\
  \end{interlude}
  
  \begin{verse}
    |: ^Komm nach Hause, Mädchen, \\
    ^Gäste warten ^dort! :| \\
    ^Bin schon auf der Brü^cke, \\
    ^bin schon auf der Brü^cke, \\
    bin schon ^zu weit ^fort! {\itshape *Md} \\
  \end{verse}

  \begin{verse}
    |: ^Warum gehst du von mir? \\
    ^Was hab ich ge^tan? :| \\
    ^Das will ich nicht sa^gen, \\
    ^das will ich nicht sa^gen, \\
    das geht ^dich nichts ^an. \\
  \end{verse}

  \begin{interlude}
   |: ^Karolinka Karo^linka \\
    ^jetz brichst du mir gar mein ^Herz! :| {\itshape *Ak} \\
  \end{interlude}
  
  \begin{verse}
    |: ^Das ist Karolinkas 
    ^allerletztes ^Wort, :| \\
    | : ^und dem hübschen Bur^schen, \\
    ^und dem hübschen Bur^schen, \\
    läuft das ^Mädchen ^fort. :| \\
  \end{verse}

  

\end{multicols}
\end{song}
\end{document}




  


 