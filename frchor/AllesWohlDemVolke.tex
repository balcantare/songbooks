\documentclass{leadsheet}
\usepackage[margin=1.5cm]{geometry}
%\usepackage{leadsheets}[bar-shortcuts=true]
\usepackage{multicol}
\usepackage{array}
\usepackage[german]{babel}
% use Helvetica
\usepackage{helvet}
\renewcommand{\familydefault}{\sfdefault}
\fontfamily{phv}\selectfont
\renewcommand{\normalsize}{\fontsize{15pt}{14pt}\selectfont}

\useleadsheetslibrary{musejazz}
\selectlanguage{german}
\DeclareTranslation{German}{leadsheets/interlude}{Zwischenspiel}


\usepackage{enumitem}
\SetLabelAlign{labelontop}%
{\strut\smash{\parbox[t]{\textwidth}{\quad#1}}}%

\makeatletter
\defineversetypetemplate {toplabel}
{%
  \ifversenamed{%
    \begin{enumerate}[align=labelontop,%
      leftmargin=1ex]%\enumerate
    }{%
    \begin{enumerate}[leftmargin=1ex]%  
    }%
  \ifobeylines%
  {%
    \setlength{\parskip}{0pt}%
    \setleadsheets{ obey-lines-parskip = \parsep }%
  }
  {}%
  \item[{\verselabel}]%
}
{\end{enumerate}}
\makeatother

\definesongtitletemplate{ourtitle}{
  %\ifsongmeasuring
      %{\section*}
      %{\section}%
      %{\songproperty{title}}
    \section*{\songproperty{title}}  
    \begingroup\footnotesize
    \begin{tabular}{
        @{}
        >{\raggedright\arraybackslash}p{.5\linewidth}
        @{}
        >{\raggedleft\arraybackslash}p{.5\linewidth}
        @{}
      }
      \ifsongproperty{interpret}
        {\GetTranslation{leadsheets/interpret}}
        {}%
      \ifsongproperty{composer}
        {%
          &
          \GetTranslation{leadsheets/composer}: %
          \printsongpropertylist{composer}{ \& }{, }{ \& }
          \ifsongproperty{lyrics}
            {
              \\ &
              \GetTranslation{leadsheets/lyrics}: %
              \printsongpropertylist{lyrics}{ \& }{, }{ \& }
            }
            {}%
        }
        {}%
      \ifsongproperty{interpret}{\\}{\ifsongproperty{composer}{\\}{}}%
      \ifsongproperty{genre}
        {& Genre: \songproperty{genre} \\}
        {}%
      \ifsongproperty{tempo}
        {& Tempo: \songproperty{tempo} \\}
        {}%
      \ifsongproperty{key}
        {%
          & \setchords{
              major = -\GetTranslation{leadsheets/major} ,
              minor = -\GetTranslation{leadsheets/minor}
            }%
          \GetTranslation{leadsheets/key}: %
          \expandcode{\writechord{\songproperty{key}}} \\%
        }
        {}%
    \end{tabular}
    \par\endgroup
  }


\begin{document}
\setleadsheets{
  align-chords={l} 
  , verse/template=toplabel
  , chorus/template=toplabel
  , chorus/label-format={\textbf}
  , interlude/template=toplabel
  , interlude/label-format={\textbf}
  % ,print-chords=false
}
\setsbfontsize{12pt}

\begin{song}[verse/numbered,remember-chords]
  { , title={Alles Wohl Dem Volke}
    , key=Dm
    , composer={Beate Tarach, Reinhard Simmgen}
  }
  \begin{multicols}{2}
  \begin{chorus}[format={\itshape}]
  'Alles ^{Dm}Wohl dem Volke!' \\
  hattet ^{A}ihr uns einst ge^{Dm}schworn. \\
  'Alles Wohl dem Volke!' \\
  doch ihr ^{C}habt das Gewissen ver^{F}lorn. \\
  'Alles ^{Gm}Wohl dem Volke!' \\
  daraus ^{Dm}wurde über Nacht, \\
  Ver^{A}rat am eignen Volke, \\
  das niemals wollte daß es jemals wieder \\
  Krieg mit^{Dm}macht. \\
  \end{chorus}
  \begin{verse}
  Ja ihr ^{A}seid die Marionetten  \\
  an den ^{Dm}Ketten der Konzerne, \\
  Woll'n die ^{A}ihre Aktien retten, \\
  helft ihr ^{Dm}ihnen dabei gerne. \\
  Und der ^{Gm}Waffenexport \\
  ist ein ^{Dm}gutbezahlter Sport. \\
  Darum ^{E}lasst die Korken knallen \\
  werft die ^{A}Skrupel über Bord! \\
  \end{verse}
  \begin{chorus}[after-label=]\end{chorus}
  \begin{verse} 
  Alles ^lasst ihr euch gefallen, \\
  als A^merikas Vasallen! \\
  Die die ^schlimmsten Kriege führen, \\
  dürfen ^Bomben stationieren, \\
  hier in ^unserem Land! \\
  Doch ihr ^habt's nicht in der Hand. \\
  Würden ^die einst detonieren, \\
  gäb's nen ^Weltenbrand! \\
  \end{verse}
  \begin{chorus}[after-label=]\end{chorus}
  \begin{verse} 
  Ja, ihr ^schwört die Menschen ein, \\
  daß sie ^denken, was sie  sollen! \\
  Denn es ^kann ja wohl nicht sein, \\
  daß sie ^machen was sie wollen. \\
  Uns're ^Demokratie \\
  ist be^droht wie noch nie. \\
  Hörn wir ^jetzt nicht auf zu schweigen, \\
  dann ver^lieren wir sie! \\
  \end{verse}  
  \begin{verse} 
  Es wird ^klar: ihr habt gelogen, \\
  habt die ^Menschen aufgehetzt! \\
  Manche ^Seele schon verbogen, \\
  manches ^Messer schon gewetzt! \\
    Ihr ^macht für Krieg bereit, \\
  doch nun ^ist es an der Zeit  \\
  für Ver^handlung statt Verderben, \\
  daß bald ^Schluß ist mit dem Leid! \\
  \end{verse}  
  \begin{chorus}[format={\itshape}]
  'Alles ^Wohl dem Volke!' \\
  nehmen ^wir's jetzt in die ^Hand! \\
  'Alles Wohl dem Volke!' \\
  und wir ^rufen es laut in das ^Land! \\
  'Alles ^Wohl dem Volke!' \\
  dafür ^sind wir bereit! \\
 'Alles ^Wohl dem Volke!' \\
  dafür kämpfen wir und für'ne bess're ^Zeit! \\
  \end{chorus}  
  \end{multicols}
\end{song}
\end{document}




  


 