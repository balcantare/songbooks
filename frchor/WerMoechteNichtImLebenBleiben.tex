\documentclass{leadsheet}
\usepackage[margin=1.5cm]{geometry}
%\usepackage{leadsheets}[bar-shortcuts=true]
\usepackage{multicol}
\usepackage{array}
\usepackage[german]{babel}
% use Helvetica
\usepackage{helvet}
\renewcommand{\familydefault}{\sfdefault}
\fontfamily{phv}\selectfont
\renewcommand{\normalsize}{\fontsize{15pt}{14pt}\selectfont}

\useleadsheetslibrary{musejazz}
\selectlanguage{german}
\DeclareTranslation{German}{leadsheets/interlude}{Zwischenspiel}


\usepackage{enumitem}
\SetLabelAlign{labelontop}%
{\strut\smash{\parbox[t]{\textwidth}{\quad#1}}}%

\makeatletter
\defineversetypetemplate {toplabel}
{%
  \ifversenamed{%
    \begin{enumerate}[align=labelontop,%
      leftmargin=1ex]%\enumerate
    }{%
    \begin{enumerate}[leftmargin=1ex]%  
    }%
  \ifobeylines%
  {%
    \setlength{\parskip}{0pt}%
    \setleadsheets{ obey-lines-parskip = \parsep }%
  }
  {}%
  \item[{\verselabel}]%
}
{\end{enumerate}}
\makeatother

\definesongtitletemplate{ourtitle}{
  %\ifsongmeasuring
      %{\section*}
      %{\section}%
      %{\songproperty{title}}
    \section*{\songproperty{title}}  
    \begingroup\footnotesize
    \begin{tabular}{
        @{}
        >{\raggedright\arraybackslash}p{.5\linewidth}
        @{}
        >{\raggedleft\arraybackslash}p{.5\linewidth}
        @{}
      }
      \ifsongproperty{interpret}
        {\GetTranslation{leadsheets/interpret}}
        {}%
      \ifsongproperty{composer}
        {%
          &
          \GetTranslation{leadsheets/composer}: %
          \printsongpropertylist{composer}{ \& }{, }{ \& }
          \ifsongproperty{lyrics}
            {
              \\ &
              \GetTranslation{leadsheets/lyrics}: %
              \printsongpropertylist{lyrics}{ \& }{, }{ \& }
            }
            {}%
        }
        {}%
      \ifsongproperty{interpret}{\\}{\ifsongproperty{composer}{\\}{}}%
      \ifsongproperty{genre}
        {& Genre: \songproperty{genre} \\}
        {}%
      \ifsongproperty{tempo}
        {& Tempo: \songproperty{tempo} \\}
        {}%
      \ifsongproperty{key}
        {%
          & \setchords{
              major = -\GetTranslation{leadsheets/major} ,
              minor = -\GetTranslation{leadsheets/minor}
            }%
          \GetTranslation{leadsheets/key}: %
          \expandcode{\writechord{\songproperty{key}}} \\%
        }
        {}%
    \end{tabular}
    \par\endgroup
  }


\begin{document}
\setleadsheets{
  align-chords={l} 
  , verse/template=toplabel
  , chorus/template=toplabel
  , chorus/label-format={\textbf}
  , interlude/template=toplabel
  , interlude/label-format={\textbf}
  % ,print-chords=false
}
\setsbfontsize{14pt}

\begin{song}[verse/numbered,remember-chords]
  { , title={Wer möchte nicht im Leben bleiben}
    , key=C
    , composer={Kurt Schwaen}
    , lyrics={Wera Küchenmeister}
  }
  \begin{multicols}{2}
    \ifsbprintchords{  
    \begin{interlude}
      \begin{tabular}[t]{@{}lllllllll}
        |_{*C}&&|&&|_{Am}&_{Dm}&|_{G}&_{C}&| \\
      \end{tabular}
    \end{interlude}
  }
  {
    \begin{interlude}[after-label=]\end{interlude}
  }
  
  \begin{verse}
    Wer ^{C}möchte ^{Dm}nicht im Leben ^{C}bleiben, \\
    die ^{G}Sonne und den Mond be^{C}sehn, \\
    |: mit ^{Dm}Winden sich umher zu ^{G}treiben \\
    und an ^{G}Wassern still zu ^{*C}stehn. :| \\
  \end{verse}

  \begin{interlude}[after-label=]\end{interlude}

  \columnbreak
  \begin{verse}
    Wer ^möchte ^nicht im Leben ^bleiben, \\
    ^Mensch' und Tieren zuge^sellt. \\
    |: Wer ^ließe sich denn gern ver^treiben, \\
    von dieser ^reichen bunten ^Welt. :| \\
  \end{verse}

  \begin{interlude}[after-label=]\end{interlude}
  
  \begin{verse}
    O ^lasset ^uns im Leben ^bleiben, \\
    weil ^jeden Tag ein Tag be^ginnt. \\
    |: O ^wollt sie nicht zu früh ver^treiben, \\
    alle ^die lebendig ^sind. :| \\
  \end{verse}

  \begin{interlude}[after-label=]\end{interlude}
\end{multicols}
\end{song}
\end{document}




  


 